\chapter{Θεωρητικό Υπόβαθρο}

\vspace{3em}

\section{Μέθοδοι Σχεδίασης Αντικειμένων}

Σαν σχέδιο, απεικόνιση ή σκηνή εννοούμε την αναπαράσταση ενός αντικειμένου πάνω σε μια επίπεδη επιφάνεια δύο διαστάσεων. Σε μια αναπαράσταση γίνεται αντιληπτή και ακριβής η μορφή του αντικειμένου, σε πραγματικό και μη - μέγεθος με κάθε λεπτομέρεια. Δεδομένης της φύσης των τριών διαστάσεων στις πραγματικές συνθήκες ενός αντικειμένου, ήταν σημαντικό να αναπτυχθούν μέθοδοι σχεδίασης και απεικόνισης των τρισδιάστατων αντικειμένων σε διδιάστατες συσκευές ή περιβάλλοντα.

Υπάρχουν δύο μέθοδοι προβολής αντικειμένων, η παράλληλη και η προοπτική. Στην παράλληλη προβολή, θεωρούμε ότι το προς σχεδίαση ή απεικόνιση αντικείμενο το βλέπουμε απο πoλύ μκαριά (άπειρο) και οι ακτίνες προβολής είναι παράλληλες μεταξύ τους. Στην προοπτική προβολή βλέπουμε το αντικείμενο από κοντά και από συγκεκριμένη θέση, έτσι οι ακτίνες προβολής δεν είναι παράλληλες, αλλά συγκλίνουν σε ένα συγκεκριμένο σημείο, το Σημείο Φυγής. Παρότι υπάρχουν επιμέρους διακρίσεις των προβολών, εμείς θα ασχοληθούμε με τις ιδιαίτερες περιπτώσεις των πλάγιων παράλληλων, Cavalier και Cabinet. Σαφέστατα κάθε κατηγορία έχει πλεονεκτήματα και μειονεκτήματα και η χρήση της καθεμίας εξαρτάται από τον σκοπό της αξιοποίησής τους, καθώς η λειτουργικότητα και η κατασκευή αποτελούν θεμελιώδη κριτήρια επιλογής των προβολών. 

\subsubsection{Πλεονεκτήματα - Μειονεκτήματα}

Μπορούμε να συμπεράνουμε ότι κάθε προβολή έχει συγκεκριμένο σκοπό, χρήση, και μπορεί να προτιμηθεί σε διαφορετικές καταστάσεις. Το βασικό χαρακτηριστικό της προοπτικής προβολής είναι πως η προσομοίωση είναι με τρόπο που το ανθρώπινο μάτι αντιλαμβάνεται το βάθος σε ένα χώρο. Γι' αυτό αν και μοιάζει ρεαλιστικό καθώς το μέγεθος μεταβάλλεται με την απόσταση, οι μετρήσεις δεν είναι ευκόλως ακριβείς. Επομένως, αν και λιγότερο ρεαλιστική, η παράλληλη προβολή μας δίνει την ευκαιρία να κάνουμε ακριβείς μετρήσεις αφού οι αποστάσεις παραμένουν αναλλοίωτες, όπως στην περίπτωση του σχεδίου για αρχιτεκτονική και έργα. 

\vspace{3em}
\begin{figure}[H]
\centering
\includegraphics[width=1.0\textwidth]{images/projection_forms.jpeg}
\caption{Διάγραμμα μεθόδων προβολής αντικειμένων}
\end{figure}




\section{Προοπτική Προβολή}

Στον πραγματικό κόσμο, ο τρόπος που αντιλαμβάνεται το ανθρώπινο μάτι τα αντικείμενα και το χώρο αφορά την προοπτική του βάθους. Αυτό μεταφράζεται ως εξής: Τα αντικείμενα που είναι πιο μακριά φαίνονται πιο μικρά. Ακόμη, οι πλεύρες ενός 3Δ αντικειμένου που είναι πιο κοντά, φαίνονται μεγαλύτερες από τις πίσω. Δεδομένου ότι στη γραφική υπολογιστών η οθόνη ερμηνεύεται ως ένα 2Δ παράθυρο σε έναν 3Δ κόσμο, πρέπει να γίνονται μετασχηματισμοί προοπτικής για την ρεαλιστική απεικόνισή τους. Σύμφωνα με τη μέθοδο εφαρμογής της προοπτικής, ο παρατηρητής ενός αντικειμένου βρίσκεται κοντά σε αυτό και τόσο η θέση όσο και η απόστασή του από αυτό είναι συγκεκριμένες. Ακόμη, η γεωμετρία της κατάστασης θέασης είναι τριγωνική, με χαρακτηριστική τη σμίκρυνση, ενώ ακόμη οι παράλληλες γραμμές έχουν κοινό ένα σημείο σύγκλισης και το κέντρο προβολής είναι ένα συγκεκριμένο σημείο. 

\section{Παράλληλη Προβολή}

Η παράλληλη προβολή διακρίνεται σε πλάγια και ορθή. Η διαφορά της με την προοπτική αφορά τις ακτίνες προβολής. Σε αυτήν την προβολή, ακριβώς επειδή είναι παράλληλη, οι ακτίνες προβολής στο επίπεδο είναι παράλληλες μεταξύ τους, δίχως να συγκλίνουν σε κοινό σημείο, και το κέντρο προβολής είναι στο άπειρο. Αυτό σημαίνει ότι τα αντικείμενα που βρίσκονται πιο μακριά δεν εμφανίζονται μικρότερα, αλλά απεικονίζονται με το πραγματικό τους μέγεθος. Έτσι εξασφαλίζεται η διατήρηση των αποστάσεων και δεν υπάρχει παραμόρφωση. Η πλάγια προβολή χωρίζεται σε δύο ειδικές περιπτώσεις: την πλάγια παράλληλη προβολή Cavalier και την πλάγια παράλληλη προβολή Cabinet.

\vspace{1em}
\begin{figure}[H]
\centering
\includegraphics[width=1.0\textwidth]{images/oblique_proj.jpeg}
\caption{Περιπτώσεις πλάγιας προβολής - Cavalier, Cabinet, Γενικευμένη}
\end{figure}

\vspace{3em}

\subsubsection{Πλάγια Προβολή Cavalier}

Στην περίπτωση της προβολής Cavalier, η γωνία προβολής είναι $45^\circ$ και η προβολή χαρακτηρίζεται ως μονομετρική. Αυτό σημαίνει ότι τα μήκη των προβολών για ένα αντικείμενο στους τρεις άξονες των συντεταγμένων $(x,y,z,)$ είναι ίσα, δηλαδή είναι 1-1-1. Παρότι οι διαστάσεις είναι πραγματικές, το αντικείμενο στη σχεδίαση ή την μοντελοποίησή του φαίνεται παραμορφωμένο. 


\subsubsection{Πλάγια Προβολή Cabinet}

Στην περίπτωση της προβολής Cabinet, η γωνία προβολής είναι συνήθως $60^\circ$, συγκεκριμένα $63,4^\circ$. Η Cabinet, σε αντίθεση με την Cavalier χαρακτηρίζεται ως διμετρική, καθώς τα μήκη των προβολών για τους άξονες $(x,y)$ είναι ίσα και $1$, ενώ το μήκος της προβολής στον άξονα $z$ είναι το μισό. Η Cabinet χρησιμοποιείται συχνότερα όταν ένα αντικείμενο έχει βάθος και θέλουμε να διατηρήσουμε την ακρίβεια και τη ρεαλιστική απεικόνιση. 

\begin{figure}[H]
\centering
\includegraphics[width=1.0\textwidth]{images/cav_cab.jpeg}
\end{figure}

