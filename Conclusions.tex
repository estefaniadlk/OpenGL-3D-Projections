\addcontentsline{toc}{chapter}{Συμπεράσματα}
\chapter*{Συμπεράσματα}
Η δυνατότητα απεικόνισης και αλληλεπίδρασης με γραφικά στοιχεία επιτρέπει στο χρήστη να έχει έναν οπτικό τρόπο αναπαράστασης και επεξεργασίας γεωμετρικών δομών. Σε συνολικό βαθμό ο κώδικας παρέχει μια ικανοποιητική βάση και ανταποκρίνεται στις απαιτήσεις, ενώ προσφέρει ευελιξία και δυνατότητες για επέκταση και προσαρμογή. Με την υπάρχουσα υλοποίηση, μπορεί να γίνει αντιληπτή η έννοια της προοπτικής και της προβολής. Είναι επίσης εφικτή η προσθήκη επιπλέον λειτουργιών και αναβαθμίσεων, όπως η ταυτόχρονη περιστροφή σε άξονες και ο μετασχηματισμός θέασης. Συνοψίζοντας, η εργασία αυτή εστίασε στην υλοποίηση ενός κώδικα για την μοντελοποίηση και προσομοίωση ενός κύβου σε τρεις προβολές: προοπτική, Cavalier, Cabinet με τη χρήση περιβάλλοντος γραφικών. Ο κώδικας επιτρέπει στο χρήστη να αλληλεπιδράσει με τα αντικείμενα επιτυχώς και ομαλά.  

\addcontentsline{toc}{chapter}{Επίλογος}
\chapter*{Επίλογος}
Mέσω αυτής της εργασίας κατανοήσαμε τις έννοιες των προβολών 3Δ αντικειμένων σε ένα 2Δ γραφικό περιβάλλον. Συνολικά αυτή η εργασία μας έδωσε την ευκαιρία να εξοικειωθούμε με τη ρεαλιστική αναπαράσταση και με αυτήν τη βάση μπορούμε να προχωρήσουμε σε πιο σύνθετες εφαρμογές όπως ένα υβριδικό μοντέλου σχεδίου χειρός και πλατφόρμας απεικόνισης για υποστήριξη αρχιτεκτονικών, μηχανολογικών ή καλλιτεχνικών έργων.
Θέτουμε στόχους αναβάθμισης της εφαρμογής μας και εστιάζουμε στις νέες προοπτικές για την ανάπτυξη και έκφραση των ιδεών μας, κάνοντας πιο ρεαλιστικά και εντυπωσιακά τα σχέδιά μας σε προγραμματιστικό περιβάλλον και επίπεδο, διευρύνοντας την αισθητική οπτική των έργων. 
