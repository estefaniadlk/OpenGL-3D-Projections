\addcontentsline{toc}{chapter}{Εισαγωγή}
\chapter*{Εισαγωγή}

Οι προβολές είναι από τα κυριότερα στοιχεία της γραφικής απεικόνισης και αφορούν την αναπαράσταση αντικειμένων. Δύο από τις προβολές που παρέχουν ρεαλιστικό αποτέλεσμα  και χρησιμοποιούνται είναι οι προβολές \textlatin{Cavalier} και \textlatin{Cabinet}, που ανήκουν στην οικογένεια των πλάγιων παράλληλων προβολών και η χρήση τους επεικτείνεται ευρέως σε πολλούς τομείς, όπως η αρχιτεκτονική, ο σχεδιασμός επίπλων και η τεχνική έργων.

Στόχος της εργασίας μας είναι η εξέταση της εφαρμογής των προβολών στην απεικόνιση ενός κύβου και η κατανοήση των πλεονεκτημάτων που προσφέρουν στην αναπαράσταση αντικειμένων. Η ευελιξία των προβολών επιτρέπει την ανάπτυξη καινοτόμων προσεγγίσεων από την αρχιτεκτονική ως την αβάν-γκαρντ έκφραση. Άλλωστε, η προοπτική ήταν σε ακμή ακόμη και από την Αναγέννηση, αν αναλογιστούμε το μεγαλειώδες έργο πρωτοποριακής απεικόνισης και εκσυγχρονισμό των εννοιών λειτουργίας και δόμησης από ζωγράφους της εποχής στα σκίτσα τους, όπως ο Λεονάρντο Ντα Βίντσι. Κατά τον Μ. Σκολάρι (1943, αρχιτέκτονας), ο ντα Βίντσι επέλεγε την παράλληλη προβολή επειδή ηταν καταλληλότερη για να απεικονίσει τον πραγματικό χώρο ενός αντικειμένου, παρά ένα αντικείμενο στο χώρο. 

\vspace{1em}

\begin{figure}[H]
\centering
\includegraphics[width=0.75\textwidth]{images/da_vinci.jpeg}
\caption{\textlatin{Leonardo da Vinci, Codex Atlanticus, 1478 - 1519}
}
\end{figure}
\pagebreak

