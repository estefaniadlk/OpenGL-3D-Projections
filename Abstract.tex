\addcontentsline{toc}{chapter}{Περίληψη}
\chapter*{Περίληψη}

Η παρούσα εργασία επιδιώκει να εξετάσει τις προβολές τρισδιάστατων αντικειμένων σε ένα επίπεδο, με έμφαση στις ειδικές περιπτώσεις των πλάγιων παράλληλων προβολών. Αρχικά, παρουσιάζονται οι κατηγορίες των προβολών, περιγράφονται οι αρχές λειτουργίας και οι χαρακτηριστικές ιδιότητες κάθε μίας από αυτές, ενώ εξετάζονται και τα πλεονεκτήματα και οι περιορισμοί που συνοδεύουν τη χρήση τους. Στη συνέχεια, παρουσιάζεται ένα πρακτικό παράδειγμα εφαρμογής των πλάγιων παράλληλων προβολών \textlatin{Cavalier} και \textlatin{Cabinet} σε έναν μοναδιαίο κύβο.
Μετά από τη θεωρητική ανάλυση, παρουσιάζεται η υλοποίηση του προγράμματος σε ένα υπολογιστικό περιβάλλον. Το πρόγραμμα δημιουργεί ένα παράθυρο που είναι χωρισμένο σε τρεις περιοχές: μία για προοπτική προβολή, μία για την προβολή \textlatin{Cavalier} και μία για την προβολή \textlatin{Cabinet}. Σε κάθε περιοχή, το πρόγραμμα σχεδιάζει έναν μοναδιαίο κύβο, ο οποίος μπορεί να περιστραφεί διαδραστικά και σε πραγματικό χρόνο γύρω από τους άξονες $x, y$ και $z$. Ο κώδικας χρησιμοποιεί μητρώα για τους μετασχηματισμούς και εντολές σχεδίασης αρχεγόνων αντικειμένων σε δύο διαστάσεις. Για την επεξεργασία και απεικόνιση του κύβου, χρησιμοποιήθηκε η βιβλιοθήκη \textlatin{OpenGL} στη γλώσσα προγραμματισμού \textlatin{C++}. Με την πρακτική εφαρμογή των προβολών \textlatin{Cavalier} και \textlatin{Cabinet} σε έναν μοναδιαίο κύβο, επιδιώκουμε να εμβαθύνουμε στην κατανόηση των εν λόγω προβολών και να εξοικειωθούμε με την αλληλεπίδρασή τους σε ένα πραγματικό περιβάλλον, θέτοντας σαν μελλοντικό στόχο την πιο ακριβή και εντυπωσιακή απεικόνιση τρισδιάστατων αντικειμένων. 
\vspace{1.5em}

\section*{Λέξεις - κλειδιά}
Προοπτική προβολή, \textlatin{Cavalier}, \textlatin{Cabinet}, Υπολογιστική Γραφική, \textlatin{OpenGL}

\newpage

\chapter*{\textlatin{Abstract}}

\selectlanguage{english} 
The subject of our project is the projections of three-dimensional objects onto a two-dimensional plane, with a specific focus on the special cases of oblique parallel projections. In the initial stage, we discuss the categories of projections, describe the operating principles and distinctive properties of each one, examine their advantages and limitations, and present a practical example of applying the oblique parallel projections, Cavalier and Cabinet, to a unit cube. Subsequently, we present the implementation of the program within a computational environment. The program creates a window divided into three areas: one for perspective projection, one for Cavalier projection, and one for Cabinet projection. In each area, the program draws a unit cube, which can be interactively rotated in real-time around the x, y, and z axes. The code utilizes matrices for transformations and drawing commands for two-dimensional primitive objects. The OpenGL library in the C++ programming language is employed for the processing and rendering of the cube. Through the practical application of Cavalier and Cabinet projections on a unit cube, our goal is to deepen our understanding of these projections and become familiar with their interaction in a real-world environment. Acquiring such knowledge and skills is crucial in the field of computer graphics and design, as it can lead us to more accurate and impressive visual representations of three-dimensional objects.

\vspace{1.5em}

\section*{\textlatin{Key Words}}
\textlatin{Perspective Projection, Cavalier, Cabinet, Computer Graphics, OpenGL}
